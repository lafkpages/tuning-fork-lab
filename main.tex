\documentclass{article}

% For positioning figures (eg. images) with [H]
\usepackage{float}

% Used for \SI and SI units
\usepackage{siunitx}

% Fancy tables innit
% https://tablesgenerator.com/latex_tables
\usepackage{booktabs}

% For \abs
\usepackage{mathtools}
\DeclarePairedDelimiter\abs{\lvert}{\rvert}%

% Swap the definition of \abs* so that it resizes the size of the
% brackets, and the starred version does not.
\makeatletter
\let\oldabs\abs
\def\abs{\@ifstar{\oldabs}{\oldabs*}}
\makeatother

% For including graphs via \includegraphics
\usepackage{graphicx}
\graphicspath{{assets}}

% Graph shortcuts
\newcommand{\graph}[2]{
  \begin{figure}[H]
    \medskip
    \centering
    \includegraphics[width=1\linewidth]{#1}
    \caption{#2}
    \medskip\label{fig:#1}
  \end{figure}
}

% Links
\usepackage{hyperref}
\hypersetup{colorlinks=true, linkcolor=blue, urlcolor=cyan, citecolor=blue}
\urlstyle{same}

\title{Tuning Fork Lab}

\begin{document}

\maketitle

\section{Raw Data Collection}

\begin{table}[H]
  \centering
  \begin{tabular}{@{}ccccccc@{}}
    \toprule
    & \multicolumn{6}{c}{Difference from the length of the cylinder
    (m)} \\
    \midrule
    Frequency (Hz) & T1    & T2    & T3    & T4    & T5    & Average \\
    256            & 0.331 & 0.334 & 0.333 & 0.336 & 0.335 & 0.3335  \\
    329            & 0.254 & 0.255 & 0.254 & 0.257 & 0.260 & 0.2550  \\
    440            & 0.192 & 0.194 & 0.193 & 0.191 & 0.196 & 0.1925  \\
    523            & 0.162 & 0.166 & 0.163 & 0.166 & 0.164 & 0.1643  \\
    1700           & 0.055 & 0.045 & 0.045 & 0.047 & 0.044 & 0.0480  \\
    \bottomrule
  \end{tabular}
  \caption{raw data collected}\label{tab:raw-data}
\end{table}

Table~\ref{tab:raw-data} shows the raw data we collected. Our
independent variable was
the frequency of the tuning fork used, and the dependent variable was
the length of the cylindrical tube that was measured to achieve that
frequency's fundamental harmonic. Controlled variables were diameter
of the tube, room temperature and pressure, person that held the
tube, and person that measured the length of the tube outside the water.

The uncertainty in length of the cylinder is due to having used a
ruler to measure, and the ruler's smallest measurement is 1
millimetre. This would usually lead to an uncertainty of 0.5
millimetres, or 0.0005 metres. However, because the tube and ruler
weren't being held perfectly still while measuring the length, the
uncertainty will be kept at 5 millimetres, or 0.005 metres.

The uncertainty in frequency will be treated as zero, because the
frequency values were written on the tuning forks with unknown
uncertainty and accuracy.

The average length of the cylinder was calculated as follows:

\[L_{avg}=\frac{L_1+L_2+L_3+L_4+L_5}{5}\]

An example calculation for frequency 256 Hertz:

\[L_{avg}=\frac{0.331+0.334+0.333+0.336+0.335}{5}\]
\[L_{avg}=\SI{0.3335}{\metre}\]

\section{Determining Length Uncertainty}

\begin{table}[H]
  \centering
  \begin{tabular}{@{}ccccccc@{}}
    \toprule
    & \multicolumn{6}{c}{Difference from the length of the cylinder
    (m)} \\ \midrule
    Frequency (Hz) & T1        & T2       & T3       & T4       & T5
    & Average    \\
    256            & 0.003     & 0.000    & 0.001    & 0.003    &
    0.002    & 0.0015     \\
    329            & 0.001     & 0.000    & 0.001    & 0.002    &
    0.005    & 0.0018     \\
    440            & 0.001     & 0.001    & 0.001    & 0.002    &
    0.004    & 0.0015     \\
    523            & 0.002     & 0.002    & 0.001    & 0.002    &
    0.000    & 0.0015     \\
    1700           & 0.007     & 0.003    & 0.003    & 0.001    &
    0.004    & 0.0036     \\ \bottomrule
  \end{tabular}
  \caption{difference from the mean}\label{tab:diff-from-mean}
\end{table}

Table~\ref{tab:diff-from-mean} shows the difference from the average
length of the cylinder
for each trial. The difference is calculated as follows:

\[d=|L_{avg}-L|\]

For example, the difference for the first trial with frequency 256.0
Hertz is calculated as follows:

\[d=|0.3335-0.331|\]
\[d=0.0025=0.003 m\]

Then, the uncertainty for the length is chosen as the highest of the
average difference from the average and the base uncertainty for
length. An example calculation for frequency of 256.0 Hertz:

% TODO: is a space required between the ± and the number?
% Check other occurrences of this in the document (Ctrl+F for pm)
\[\Delta L=\pm\SI{0.005}{\metre}\]

\section{Frequency vs. Length}\label{sec:freq-vs-length}

\begin{table}[H]
  \centering
  \begin{tabular}{@{}ccc@{}}
    \toprule
    Frequency (Hz) & Average length (m) & Uncertainty in average
    length (m) \\
    \midrule
    256 & 0.334 & 0.005 \\
    329 & 0.255 & 0.005 \\
    440 & 0.193 & 0.005 \\
    523 & 0.164 & 0.005 \\
    1700 & 0.048 & 0.005 \\
    \bottomrule
  \end{tabular}
  \caption{frequency vs.\ average cylinder length and uncertainty in
  average cylinder length}\label{tab:freq-vs-length}
\end{table}

\graph{freq-vs-length}{frequency vs.\ average cylinder length}

In figure~\ref{fig:freq-vs-length}, there is a clear inversely proportional
relationship between frequency and length of the cylinder. This can
also be seen in the equation of the trend line, which is in the form
of \(y=ax^{-b}\), which can also be rewritten as \(y=\frac{a}{x^b}\).

To make the relationship linear, we can graph the inverse frequency
vs.\ the length of the cylinder. However, we know that inverse
frequency in Hertz is the same as period in seconds.
% TODO: capitalise "Hertz"?

\section{Period vs. Length}

\begin{table}[H]
  \centering
  \begin{tabular}{@{}ccc@{}}
    \toprule
    Period (s) & Average length (m) & Uncertainty in average length (m) \\
    \midrule
    0.0039 & 0.334 & 0.005 \\
    0.0030 & 0.255 & 0.005 \\
    0.0023 & 0.193 & 0.005 \\
    0.0019 & 0.164 & 0.005 \\
    0.0006 & 0.048 & 0.005 \\
    \bottomrule
  \end{tabular}
  \caption{frequency vs.\ average cylinder length and uncertainty in
  average cylinder length}\label{tab:period-vs-length}
\end{table}

The period was calculated as follows, where \(T\) represents period and \(f\)
represents frequency:

\[T=\frac{1}{f}\]

So, for example, the period for a frequency of 256 Hertz is
calculated as follows:

\[T=\frac{1}{\SI{256}{\hertz}}=\SI{0.0039}{\second}\]

Period also has no uncertainty because it's calculated in terms of
frequency, which has no uncertainty.

\graph{period-vs-length}{period vs.\ length of the cylinder}

Figure~\ref{fig:period-vs-length} shows a graph of period vs.\ length
of the cylinder. The relationship is linear, as explained in
part~\ref{sec:freq-vs-length}. The graph supports the theory, as the
trend line is between all error bars. Furthermore, no points seem to
be outliers. We can use the best, minimum and maximum slopes of the
trend lines to calculate our theoretical speed of sound.

\section{Calculating Theoretical Speed of Sound}\label{sec:theoretical-speed}

We are given the equation \(L+e=\frac{v}{4f}\), where \(L\) represents cylinder
length, \(e\) represents end correction, \(v\) represents the speed of sound,
and \(f\) represents frequency. If we substitute \(f=\frac{1}{T}\) to get
an equation for length in terms of period, we get the following:

\[L+e=\frac{vT}{4}\]

If we continue rearranging:

\[L=\frac{vT}{4}-e\]
\[L=\frac{v}{4}T-e\]

In terms of figure~\ref{fig:period-vs-length}, since \(L\) is our
y-axis and \(T\) is our x-axis, this equation is now in linear form.
Therefore, our slope is \(\frac{v}{4}\) and \(-e\) is our
y-intercept. We can use this to solve for the speed of sound, since
we know the slopes from the trend lines we get from Excel in
figure~\ref{fig:period-vs-length}:

\[m=\frac{v}{4}\]
\[4m=v\]
\[v=4m\]

We can now plug in our slopes:

\[v_{best}=4\times85.443=\SI{341.772}{\metre\per\second}\]
\[v_{min}=4\times83.032=\SI{332.128}{\metre\per\second}\]
\[v_{max}=4\times89.059=\SI{356.236}{\metre\per\second}\]

And to calculate uncertainty in \(v\):

\[\Delta v=\frac{v_{max}-v_{min}}{2}\]
\[\Delta v=\frac{356.236-332.128}{2}=\SI{12.054}{\metre\per\second}\]

However, uncertainties must be rounded to one significant figure:

\[\Delta v=\pm\SI{10}{\metre\per\second}\]

That would make a final theoretical value for the speed of sound of
\(\SI{341.772}{\metre\per\second}\pm\SI{10}{\metre\per\second}\).
However, the value can't have more significant figures than the
uncertainty, so we get a final theoretical value of
\(\SI{340}{\metre\per\second}\pm\SI{10}{\metre\per\second}\).

The speed of sound in air according to Wikipedia is
\SI{343}{\metre\per\second}. Percent error can be calculated as follows:

\[\%~error=\abs{\frac{\mathrm{experimental}-\mathrm{exact}}{\mathrm{exact}}}\]
\[=\abs{\frac{340-343}{343}}\]
\[=\SI{8.74}{\percent}=\SI{9}{\percent}\]

\section{Calculating Theoretical End Correction}

As seen from the equation mentioned in
part~\ref{sec:theoretical-speed}, the y-intercept would give us our
negative end correction. Since Excel gives us the following equations:

\[y_{best}=85.443x-0.0016\]
\[y_{min}=83.032x+0.0042\]
\[y_{max}=89.059b-0.0094\]

Therefore:

\[e_{best}=\SI{0.0016}{\metre}=\SI{1.6}{\milli\metre}\]
\[e_{min}=\SI{-0.0042}{\metre}=\SI{-4.2}{\milli\metre}\]
\[e_{max}=\SI{0.0094}{\metre}=\SI{9.4}{\milli\metre}\]

To calculate uncertainty in \(e\):

\[\Delta e=\frac{e_{max}-e_{min}}{2}\]
\[\Delta e=\frac{0.0094+0.0042}{2}=\SI{0.0068}{\metre}\]
\[\Delta e=6.8 mm=7 mm\]

Therefore, our final theoretical end correction value is
\(\SI{2}{\milli\metre}\pm\SI{7}{\milli\metre}\).

According to the report provided on Google Classroom by Mr.\ George,
end correction can also be calculated as \(e=0.33D\), where \(D\)
represents the diameter of the cylindrical tube. We recorded this as
\(D=\SI{25.3}{\milli\metre}\pm\SI{0.1}{\milli\metre}\). The uncertainty
comes from the calipers used to measure the diameter of the tube. We
used Mr.\ George's digital calipers, which have a precision of 0.1 millimetres.

Using this measurement of tube diameter, we can calculate our
expected end correction value:

\[e_{expected}=0.33\times25.3=\SI{8.349}{\milli\metre}\]

And with this, we can calculate our percent error for end correction,
using the same formula for percent error as used for \(v\):

\[\%~error=\abs{\frac{2-8.349}{8.349}}=\SI{76}{\percent}\]

\section{Evaluation}

For speed of sound, a percent error of \SI{9}{\percent} is
reasonable. \SI{9}{\percent} of \SI{343}{\metre\per\second} is only
\SI{3}{\metre\per\second}, which relative to the speed of sound is very low.

For end correction, even though \SI{76}{\percent} error is very high,
we must keep in mind that it is on a small value---\SI{76}{\percent} of
our expected value is only about 6.3 millimetres. Therefore, such a
high percent error is justified.

One possible error that could have affected the accuracy of results
is the imprecision of the length of the tube measurements. Since the
tube was held by a human, it wasn't held perfectly still. And
meanwhile, another human had to hold up a ruler which was also not
perfectly still to measure the length of this non-still tube. For
example, looking at table 2, we can see that the measurements for the
tuning fork of frequency 1700 Hertz were the most imprecise:

\[L_{min}=\SI{0.044}{\metre}=\SI{44}{\milli\metre}\]
\[L_{max}=\SI{0.055}{\metre}=\SI{55}{\milli\metre}\]

The measurements had a range of 11 millimetres, or 1.1 centimetres.
That is high imprecision.

Our controlled variables were kept constant for the most part: the
diameter of the tube was constant as the same tube was used
throughout the lab. Room temperature and pressure were unlikely to
change during the approximate 15 minutes that it took us to do the
experiment, but they were also kept constant via thermostat. The
person that held the tube (Luis) and the person that measured the
lengths of the tubes outside the water (Michael) were kept constant
throughout the lab.

Five trials were chosen to spread out random error in the
measurements of the lengths of the tubes, as the more trials that are
done the more accurate average we would get, with a lower
uncertainty. We chose five trials because they were enough to have a
low enough uncertainty, and few enough to not take too long to complete.

\end{document}
